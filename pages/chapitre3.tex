\chapter{Les principes d'évolutivité du SI: SOA, API, Micro-services}

\section*{Quizz}
\addcontentsline{toc}{section}{Quizz}

\begin{enumerate}
    \item Expliquez en quoi l’évolutivité d’un système d’information est essentielle pour une entreprise en croissance.

        Cela permet à une entreprise en croissance d’ajuster ses capacités technologiques sans perturber ses opérations, en intégrant facilement de nouveaux services et en répondant à la demande accrue.
    \item Décrivez les caractéristiques principales d'une architecture orientée services (SOA) et comment elle facilite l'évolutivité.

        SOA se base sur des services réutilisables et indépendants. Cela facilite l’évolutivité en permettant d'ajouter, modifier ou remplacer des services sans affecter l'ensemble du système.
    \item Quelles sont les différences entre une API REST et une API SOAP, et dans quels cas serait-il préférable d’utiliser l'une plutôt que l’autre ?

        REST est léger, utilise des formats comme JSON, et est adapté aux applications web. SOAP est plus rigide, orienté XML, avec plus de sécurité, idéal pour des transactions complexes. REST est préféré pour sa simplicité et rapidité, SOAP pour la fiabilité dans des environnements sécurisés.
    \item Comment l’architecture des micro-services diffère-t-elle d’une architecture monolithique et quels sont les avantages pour une entreprise ?

        Les microservices découpent les fonctionnalités en services indépendants, facilitant la scalabilité et les mises à jour individuelles, contrairement à l’architecture monolithique où toutes les fonctions sont interconnectées.
    \item Quels sont les défis que peut rencontrer une entreprise lors de la mise en place d’une architecture de microservices?

        Coordination entre services, gestion de la communication interservices, et complexité de la sécurité sont parmi les principaux défis.
    \item Comment les API contribuent-elles à l’interopérabilité entre les systèmes internes et externes d’une entreprise ? Donnez des exemples d’utilisation concrète.

        Les API permettent à différents systèmes internes et externes de communiquer efficacement. Exemple : une API de paiement qui connecte un site e-commerce à un service bancaire externe.
    \item Quels sont les rôles des API dans une stratégie d'évolutivité du SI et comment assurent-elles une connexion fluide entre différents services ?

        Les API permettent l'intégration rapide de nouveaux services ou fonctionnalités, assurant la communication fluide entre modules et applications au sein du SI.
    \item Expliquez comment la modularité dans une architecture SOA aide à minimiser les risques liés aux mises à jour et à l’intégration de nouveaux services.

        La modularité permet d’isoler les services, minimisant ainsi les risques d’interruption lors des mises à jour ou de l’intégration de nouveaux services.
    \item Quels mécanismes de sécurité doivent être pris en compte lors de la mise en place d’API dans une architecture évolutive ?

        Il faut prendre en compte l’authentification (OAuth), l’autorisation, la gestion des certificats et le cryptage des données pour sécuriser les communications API.
    \item Donnez un exemple concret d'une entreprise ou d’un secteur qui bénéficierait d’une architecture basée sur des micro-services, et expliquez pourquoi.

        Netflix utilise une architecture micro-services pour gérer efficacement des services indépendants comme la diffusion vidéo, la facturation, et les recommandations, permettant une grande flexibilité et évolutivité.

\end{enumerate}
