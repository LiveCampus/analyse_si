\chapter{Introduction à la gouvernance des données}

\section*{Quizz}
\addcontentsline{toc}{section}{Quizz}
Mise en place des différents réseaux :

\begin{enumerate}
    \item Qu'est-ce que la gouvernance des données et pourquoi est-elle cruciale pour une entreprise ?

        C'est l'ensemble des pratiques et processus qui garantissent la gestion efficace des données au sein d'une organisation. Elle est cruciale pour assurer l'intégrité, la sécurité et la conformité des données, facilitant ainsi des décisions basées sur des informations fiables.
    \item Quels sont les objectifs principaux de la gouvernance des données ?

        \begin{itemize}
            \item Assurer la qualité et l'intégrité des données.
            \item Garantir la sécurité et la confidentialité des données.
            \item Faciliter l'accès aux données pertinentes.
            \item Assurer la conformité avec les réglementations.
        \end{itemize}
    \item Quelle est la différence entre qualité des données et sécurité des données ?

        La qualité des données c'est l'exactitude, la pertinence et la fiabilité des données, et la sécurité des données, plutôt la protection des données contre l'accès non autorisé et les cybermenaces.
    \item Qui sont les acteurs principaux dans la gouvernance des données ?

        \begin{itemize}
            \item Data Owners (propriétaires de données)
            \item Data Stewards (gérants de données)
            \item Data Governance Council (conseil de gouvernance des données)
            \item Utilisateurs finaux (employés utilisant les données)
        \end{itemize}
    \item Pourquoi est-il important de définir des droits d'accès aux données ?

        Car ça permet de protéger les données sensibles, de garantir la conformité aux réglementations et de réduire les risques d'erreurs ou d'abus lors de l'utilisation des données.
    \item Quelle est la fonction d'un Data Steward ?

        Un Data Steward est responsable de la gestion et de la qualité des données dans l'organisation, veillant à ce que les normes de gouvernance soient respectées et que les données soient utilisées de manière appropriée.
    \item Comment la gouvernance des données aide-t-elle à se conformer aux réglementations telles que le RGPD ?

        La gouvernance des données établit des protocoles pour la collecte, le traitement et le stockage des données personnelles, garantissant ainsi la transparence, le consentement et la protection des données, ce qui est essentiel pour la conformité au RGPD.
    \item Citez deux avantages de la gouvernance des données pour une organisation.

        \begin{itemize}
            \item Meilleure prise de décision grâce à des données fiables.
            \item Réduction des risques de non-conformité et de violations de données.
        \end{itemize}
    \item Quelles sont les différentes étapes du cycle de vie des données ?

        \begin{itemize}
            \item Collecte
            \item Stockage
            \item Traitement
            \item Analyse
            \item Archivage
            \item Suppression
        \end{itemize}
    \item Quelles sont les conséquences possibles d'une mauvaise gouvernance des données dans une entreprise ?

        \begin{itemize}
            \item Perte de confiance des clients.
            \item Amendes et sanctions pour non-conformité.
            \item Mauvaises décisions basées sur des données inexactes.
            \item Risques accrus de cyberattaques et de fuites de données.
        \end{itemize}
\end{enumerate}

\section*{Mini cas n°1}
\addcontentsline{toc}{section}{Mini cas n°1}

\textbf{Contexte}

Une entreprise de services financiers rencontre des problèmes liés à la qualité de ses données clients. Certains doublons existent, les informations ne sont pas toujours à jour, et plusieurs services utilisent des bases de données différentes avec des informations incohérentes. De plus, la sécurité des données est insuffisante, et l’entreprise fait face à des risques de non-conformité au RGPD.

\textbf{Tâches}

A partir du contexte ci-dessus, et en tant que responsable de la DSI, je vous demande de me proposer des solutions pour mettre en place une stratégie de gouvernance des données afin de résoudre les problèmes mentionnés. Indiquez les actions que vous recommanderiez par ordre de priorité.

\begin{enumerate}
    \item Évaluation de la situation actuelle

        \textbf{Action} : Réaliser un audit des données existantes.

        \begin{itemize}
            \item \textbf{Description} : Identifier toutes les sources de données, les types d'informations collectées, les systèmes en place et les processus de gestion de données.
            \item \textbf{Résultat attendu} : Une cartographie des données et une compréhension claire des doublons, des incohérences et des lacunes.
        \end{itemize}

    \item Définition d’une politique de gouvernance des données

        \textbf{Action} : Établir une politique de gouvernance des données claire.

        \begin{itemize}
            \item \textbf{Description} : Documenter les principes directeurs, les rôles et responsabilités, ainsi que les processus pour la gestion, la qualité et la sécurité des données.
            \item \textbf{Résultat attendu} : Un cadre solide qui guide les décisions et actions relatives aux données.
        \end{itemize}

    \item Mise en place d'un référentiel de données

        \textbf{Action} : Créer un référentiel centralisé de données clients.

        \begin{itemize}
            \item \textbf{Description} : Utiliser un système de gestion de la qualité des données (DQMS) pour centraliser les données clients et assurer la synchronisation entre les différents services.
            \item \textbf{Résultat attendu} : Réduction des doublons et accès cohérent aux données clients pour tous les services.
        \end{itemize}

    \item Normalisation et nettoyage des données

        \textbf{Action} : Lancer un programme de nettoyage et de normalisation des données.

        \begin{itemize}
            \item \textbf{Description} : Identifier et supprimer les doublons, mettre à jour les informations obsolètes et appliquer des standards de formatage cohérents.
            \item \textbf{Résultat attendu} : Amélioration de la qualité des données, rendant les informations plus fiables et exploitables.
        \end{itemize}

    \item Renforcement de la sécurité des données

        \textbf{Action} : Mettre en œuvre des mesures de sécurité des données robustes.

        \begin{itemize}
            \item \textbf{Description} : Évaluer les vulnérabilités existantes et installer des systèmes de protection (chiffrement, contrôle d'accès, audits réguliers).
            \item \textbf{Résultat attendu} : Protection des données sensibles et réduction des risques de fuites d’informations.
        \end{itemize}

    \item Formation et sensibilisation des employés

        \textbf{Action} : Former le personnel sur la gouvernance des données et la conformité au RGPD.

        \begin{itemize}
            \item \textbf{Description} : Mettre en place des sessions de formation régulières sur l'importance de la qualité des données, la sécurité et le respect de la réglementation.
            \item \textbf{Résultat attendu} : Culture d'entreprise axée sur la responsabilité et la protection des données.
        \end{itemize}

    \item Suivi et évaluation continue

        \textbf{Action} : Instaurer un processus de suivi et d'évaluation de la qualité des données.

        \begin{itemize}
            \item \textbf{Description} : Mettre en place des indicateurs de performance (KPI) pour surveiller la qualité des données et l’efficacité des politiques mises en œuvre.
            \item \textbf{Résultat attendu} : Amélioration continue des processus de gestion des données et capacité d’adaptation aux changements réglementaires.
        \end{itemize}

    \item Mise en conformité avec le RGPD

        \textbf{Action} : Assurer la conformité avec le RGPD.

        \begin{itemize}
            \item \textbf{Description} : Évaluer les pratiques de gestion des données actuelles, identifier les lacunes en matière de conformité et établir des processus pour le respect des droits des utilisateurs (accès, rectification, suppression des données).
            \item \textbf{Résultat attendu} : Réduction des risques juridiques et amélioration de la confiance des clients.
        \end{itemize}

\end{enumerate}
